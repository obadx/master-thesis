% Settings for Preliminary Pages
\cleardoublepage
\newpage
\thispagestyle{empty}
\phantomsection
\addtotoc{\arb{الملخص}}

%\pagestyle{plain} % No headers, just page numbers
%\setcounter{page}{2}

%-----------------------------------------

%-----------------------------------------
\begin{center}\huge \textbf{\arb{الملخص}}\end{center}

\begin{center}

\arb{

أنزل الله كتابه وتعهد بحفظه ويسره للذكر ولذلك تعاقب العلماء المسلمون على العناية بالقرآن من جميع الجوانب من اللفظ حتى المعنى وطوعوا الوسائل التقنية المتاحة في زمنهم لخدمة القرآن الكريم. وفي عصر الذكاء الاصطناعي نحاول تقريب القرآن من المسلمين عن طريق تقديم طريقة مبتكرة لكشف وتصحيح أخطاء التلاوة والتجويد وصفات الحروف لدى متعلمي القرآن الكريم عن طريق تدقيم الرسم الصوتي القرآن الكريم متعدد المستويات للحروف وصفاتها ولقلة وفرة التلاوات القرآنية المخصصة لتدريب نماذج الذكاء الاصطناعي قدمنا طريقة شبه مأتمتة بنسبة 98\% لبناء قواعد بيانات قرآنية عالية الجوة و قدمنا أيضا قاعدة بيانات مكنونة من 890 ساعة تحتوي على 300 ألف عينة مُعلَّمة وبجانب ذلك قدمنا نموذج مبتكر CTC متعدد المستويات وبفضل الله حققنا متوسط معدل خطأ حرفي قدره 16\%. مما يؤكد دقة الرسم الصوتي للقرآن الكريم وسهولة تعلمه ويؤكد تلك الحقيقة ألا وهي: ﴿ وَلَقَدْ يَسَّرْنَا ٱلْقُرْءَانَ لِلذِّكْرِ فَهَلْ مِن مُّدَّكِرٍ ﴾

}
\end{center}

%\begin{center}
%\underline{\textbf{Summary}}
%\end{center}
%



