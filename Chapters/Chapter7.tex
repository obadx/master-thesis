% Chapter Template

\chapter{Conclusion} % Main chapter title

\label{Chapter7} % Change X to a consecutive number; for referencing this chapter elsewhere, use \ref{ChapterX}

\lhead{Chapter 7. \emph{Conclusion}} % Change X to a consecutive number; this is for the header on each page - perhaps a shortened title

\section*{Conclusion}

We present a new way of assessing pronunciation errors of Holy Quran learners by developing multi-level Quran Phonetic Script capable of capturing all pronunciation errors for \arb{حفص} except for \arb{إشمام} (as it is a sign not pronounced by mouth), along with 890 hours and 300K of annotated data, a 98\% pipeline to create similar data, plus modeling and validation \cite{citation}.


\section{Limitations}

Our primary limitation is that our dataset consists of golden recitations with no errors, limiting our ability to evaluate performance on real-world data. Although we tested on a few actual samples and successfully detected \arb{مد}, \arb{غنة}, and \arb{قلقة} errors, we need to develop a comprehensive dataset containing error-containing recitations transcribed with our Quran Phonetic Script.

A secondary limitation arises from attribute-specific articulation patterns: Certain attributes apply exclusively to individual letters, such as `Istitala` for (\arb{ض}) and `Tikrar` for (\arb{ر}). Consequently, we expect our model will be unable to capture instances of (\arb{ض}) without `Istitala` or (\arb{ر}) without `Tikrar`. This limitation similarly applies to Tajweed rules that occur less frequently in the Holy Quran, such as \arb{إمالة}, \arb{روم}, and \arb{تسهيل}.

\section{Future Work}

To address these limitations, we plan to:

\begin{enumerate}
\item \textbf{Develop an Error-Included Dataset:}
Collect and annotate a large-scale dataset of learner recitations containing common Tajweed errors, transcribed using our phonetic script. This will enable more robust model training and evaluation.

\item \textbf{Expand to Other Recitation Styles (\arb{روايات}):}
Extend the phonetic script and modeling framework to support additional recitation styles (e.g., \arb{ورش} or \arb{قالون}), facilitating broader applicability across the Muslim world.

\item \textbf{Deploy and Evaluate in Real-World Settings:}
Integrate the model into user-friendly applications and evaluate its effectiveness in real-world learning environments, incorporating feedback from Quran teachers and students to iteratively improve the system.
\end{enumerate}

By addressing these challenges, we aim to advance the state of Quranic pronunciation assessment and make automated, accurate feedback accessible to learners worldwide.

